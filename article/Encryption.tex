% My Encryption Algorithim Article
% Henry J Schmale
% March 3, 2015

\documentclass[12pt,draft]{article}

% Margins
\usepackage[letterpaper]{geometry}
\geometry{
  top    = 1.0in,
  bottom = 1.0in,
  left   = 1.0in,
  right  = 1.0in}

% Font
\usepackage{times}

% Spacing
\setlength{\textheight}{9.5in}

\usepackage{mathtools}

% Start of the document
\begin{document}

\section{Introduction}
  Encryption is a powerful tool to protect your information, but standard
  encryption can be insecure if found. Additionally, it can be incriminating
  if found, so that is why hiding the encrypted information is more important
  then ever. This practice is called stenography, and I have developed such
  a system to encrypt a text document, and hide it inside of an image, this
  document describes in detail the processed used to encrypt and hide the
  information inside of a photograph.

\section{Encryption and Decryption}
  The encryption used in this system is a simple XOR rotate cipher.
  The cipher uses an n-byte key to encrypt the message, and
  throughtout the encryption of the message the key changes. The
  changes in the key over the process of encryption can be described
  by the equation below.
  
  % Equation describing the value of K to use in this iteration
  \begin{equation}
    \label{eq:keyDerive}
    K_i = rightRotate(
        \underbrace{K_i \bmod length(K)}_\text{Value to rotate},
        \underbrace{\frac{i}{length(K)} \bmod (sizeof(type(K)) 
            \times B)}_\text{places}) 
  \end{equation}

  The above equation describes how to calculate the key value to use
  in iteration \( i \) of the encryption algorithim. 

  

\end{document}
