% My Encryption Algorithim Article
% Henry J Schmale
% March 3, 2015

\documentclass[12pt,draft]{article}

% Margins
\usepackage[letterpaper]{geometry}
\geometry{
  top    = 1.0in,
  bottom = 1.0in,
  left   = 1.0in,
  right  = 1.0in}

% Font
\usepackage{times}

% Spacing
\setlength{\textheight}{9.5in}

\usepackage{mathtools}

% Start of the document
\begin{document}

\section{Introduction}
  Encryption is a powerful tool to protect your information, but standard
  encryption can be insecure if found. Additionally, it can be incriminating
  if found, so that is why hiding the encrypted information is more important
  then ever. This practice is called stenography, and I have developed such
  a system to encrypt a text document, and hide it inside of an image, this
  document describes in detail the processed used to encrypt and hide the
  information inside of a photograph.

\section{Encryption and Decryption}
  The encryption used in this system is a simple XOR rotation cipher.
  The cipher uses an n-byte key to encrypt and decrypt the message.
  Along with the use of a rotating key, this a relatively fast and
  secure algorithim, though probably nowhere near as secure as any of
  public-private key algorithims. However, this algorithim was not
  designed with that in mind, this algorithim was designed with the
  thought of protecting a file against a cursory overview.
  
  As this algorithim uses a rotating key to encrypt mesages there
  must be regular pattern to how the key changes. As such the
  change to the key is a simple bit rotation after each use. 
  
  % Equation describing the value of K to use in this iteration
  \begin{equation}
    \label{eq:keyDerive}
    K_i = rightRotate(
        \underbrace{K_i \bmod length(K)}_\text{Value to rotate},
        \underbrace{\frac{i}{length(K)} \bmod (sizeof(type(K)) 
            \times B)}_\text{places}) 
  \end{equation}

  The above equation describes how to calculate the key value to use
  in iteration \( i \) of the encryption algorithim. It requires the use
  of a bitwise rotate operation so that no data is lost, through the
  replacement by zeros when using standard bitwise shifts. And as zeros
  were introduced into the key, the xor shift would no longer work as
  all the bits would remain the same.

  This algorithim can use an n-bit key as the new key value is
  determined indepently of the length of the key and provides
  facilities for working with various key lengths. As such it can use
  a n-bit key for encryption and decryption.

  The encryption algorithim requires the use of the bitwise xor and
  takes advantage of the speed yielded with an xor opperation, due to
  no carry bits being required, and dedicated circuitry existing on most
  modern processors. Below is a description of the algorithim in detail.

  

\end{document}
